%% Generated by Sphinx.
\def\sphinxdocclass{report}
\documentclass[letterpaper,10pt,english]{sphinxmanual}
\ifdefined\pdfpxdimen
   \let\sphinxpxdimen\pdfpxdimen\else\newdimen\sphinxpxdimen
\fi \sphinxpxdimen=.75bp\relax

\PassOptionsToPackage{warn}{textcomp}
\usepackage[utf8]{inputenc}
\ifdefined\DeclareUnicodeCharacter
% support both utf8 and utf8x syntaxes
  \ifdefined\DeclareUnicodeCharacterAsOptional
    \def\sphinxDUC#1{\DeclareUnicodeCharacter{"#1}}
  \else
    \let\sphinxDUC\DeclareUnicodeCharacter
  \fi
  \sphinxDUC{00A0}{\nobreakspace}
  \sphinxDUC{2500}{\sphinxunichar{2500}}
  \sphinxDUC{2502}{\sphinxunichar{2502}}
  \sphinxDUC{2514}{\sphinxunichar{2514}}
  \sphinxDUC{251C}{\sphinxunichar{251C}}
  \sphinxDUC{2572}{\textbackslash}
\fi
\usepackage{cmap}
\usepackage[T1]{fontenc}
\usepackage{amsmath,amssymb,amstext}
\usepackage{babel}



\usepackage{times}
\expandafter\ifx\csname T@LGR\endcsname\relax
\else
% LGR was declared as font encoding
  \substitutefont{LGR}{\rmdefault}{cmr}
  \substitutefont{LGR}{\sfdefault}{cmss}
  \substitutefont{LGR}{\ttdefault}{cmtt}
\fi
\expandafter\ifx\csname T@X2\endcsname\relax
  \expandafter\ifx\csname T@T2A\endcsname\relax
  \else
  % T2A was declared as font encoding
    \substitutefont{T2A}{\rmdefault}{cmr}
    \substitutefont{T2A}{\sfdefault}{cmss}
    \substitutefont{T2A}{\ttdefault}{cmtt}
  \fi
\else
% X2 was declared as font encoding
  \substitutefont{X2}{\rmdefault}{cmr}
  \substitutefont{X2}{\sfdefault}{cmss}
  \substitutefont{X2}{\ttdefault}{cmtt}
\fi


\usepackage[Bjarne]{fncychap}
\usepackage{sphinx}

\fvset{fontsize=\small}
\usepackage{geometry}


% Include hyperref last.
\usepackage{hyperref}
% Fix anchor placement for figures with captions.
\usepackage{hypcap}% it must be loaded after hyperref.
% Set up styles of URL: it should be placed after hyperref.
\urlstyle{same}

\addto\captionsenglish{\renewcommand{\contentsname}{Contents:}}

\usepackage{sphinxmessages}
\setcounter{tocdepth}{1}



\title{GcSim Documentation}
\date{Dec 17, 2020}
\release{0.1}
\author{University of Campania}
\newcommand{\sphinxlogo}{\vbox{}}
\renewcommand{\releasename}{Release}
\makeindex
\begin{document}

\pagestyle{empty}
\sphinxmaketitle
\pagestyle{plain}
\sphinxtableofcontents
\pagestyle{normal}
\phantomsection\label{\detokenize{index::doc}}



\chapter{Indices and tables}
\label{\detokenize{index:indices-and-tables}}\begin{itemize}
\item {} 
\DUrole{xref,std,std-ref}{genindex}

\item {} 
\DUrole{xref,std,std-ref}{modindex}

\item {} 
\DUrole{xref,std,std-ref}{search}

\end{itemize}
\phantomsection\label{\detokenize{index:module-simulator}}\index{module@\spxentry{module}!simulator@\spxentry{simulator}}\index{simulator@\spxentry{simulator}!module@\spxentry{module}}

\section{simulator}
\label{\detokenize{index:simulator}}
This  module allows for starting, stopping and checking the status of the simulator and eventually of the optimizer
\index{start() (in module simulator)@\spxentry{start()}\spxextra{in module simulator}}

\begin{fulllineitems}
\phantomsection\label{\detokenize{index:simulator.start}}\pysiglinewithargsret{\sphinxcode{\sphinxupquote{simulator.}}\sphinxbfcode{\sphinxupquote{start}}}{\emph{\DUrole{n}{args}}}{}
It starts  the first simulator and eventually the optimizer
:param args: command line arguments
:return:

\end{fulllineitems}

\index{start\_optimizer() (in module simulator)@\spxentry{start\_optimizer()}\spxextra{in module simulator}}

\begin{fulllineitems}
\phantomsection\label{\detokenize{index:simulator.start_optimizer}}\pysiglinewithargsret{\sphinxcode{\sphinxupquote{simulator.}}\sphinxbfcode{\sphinxupquote{start\_optimizer}}}{\emph{\DUrole{n}{optimizer}}, \emph{\DUrole{n}{policy}}}{}~\begin{quote}\begin{description}
\item[{Parameters}] \leavevmode\begin{itemize}
\item {} 
\sphinxstyleliteralstrong{\sphinxupquote{optimizer}} \textendash{} the optimizer to start (should be supported dummy, eurecat, oslo)

\item {} 
\sphinxstyleliteralstrong{\sphinxupquote{policy}} \textendash{} the optimization policy (Es: cheapest, greenest, earliest)

\end{itemize}

\item[{Returns}] \leavevmode
None

\end{description}\end{quote}

\end{fulllineitems}

\index{start\_simulator() (in module simulator)@\spxentry{start\_simulator()}\spxextra{in module simulator}}

\begin{fulllineitems}
\phantomsection\label{\detokenize{index:simulator.start_simulator}}\pysiglinewithargsret{\sphinxcode{\sphinxupquote{simulator.}}\sphinxbfcode{\sphinxupquote{start\_simulator}}}{\emph{\DUrole{n}{args}}}{}
It will start the agents based simulator
:return:

\end{fulllineitems}

\index{stop() (in module simulator)@\spxentry{stop()}\spxextra{in module simulator}}

\begin{fulllineitems}
\phantomsection\label{\detokenize{index:simulator.stop}}\pysiglinewithargsret{\sphinxcode{\sphinxupquote{simulator.}}\sphinxbfcode{\sphinxupquote{stop}}}{\emph{\DUrole{n}{args}}}{}
It stops both the simulator and the optimizer
:param args:
:return:

\end{fulllineitems}

\index{stop\_simulator() (in module simulator)@\spxentry{stop\_simulator()}\spxextra{in module simulator}}

\begin{fulllineitems}
\phantomsection\label{\detokenize{index:simulator.stop_simulator}}\pysiglinewithargsret{\sphinxcode{\sphinxupquote{simulator.}}\sphinxbfcode{\sphinxupquote{stop\_simulator}}}{}{}
It will stop the simulator component
:return:

\end{fulllineitems}

\phantomsection\label{\detokenize{index:module-starter}}\index{module@\spxentry{module}!starter@\spxentry{starter}}\index{starter@\spxentry{starter}!module@\spxentry{module}}\phantomsection\label{\detokenize{index:module-setupmodule}}\index{module@\spxentry{module}!setupmodule@\spxentry{setupmodule}}\index{setupmodule@\spxentry{setupmodule}!module@\spxentry{module}}

\renewcommand{\indexname}{Python Module Index}
\begin{sphinxtheindex}
\let\bigletter\sphinxstyleindexlettergroup
\bigletter{s}
\item\relax\sphinxstyleindexentry{setupmodule}\sphinxstyleindexpageref{index:\detokenize{module-setupmodule}}
\item\relax\sphinxstyleindexentry{simulator}\sphinxstyleindexpageref{index:\detokenize{module-simulator}}
\item\relax\sphinxstyleindexentry{starter}\sphinxstyleindexpageref{index:\detokenize{module-starter}}
\end{sphinxtheindex}

\renewcommand{\indexname}{Index}
\printindex
\end{document}